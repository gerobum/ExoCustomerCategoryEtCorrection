\documentclass[a4paper,10pt]{article}
\usepackage[utf8x]{inputenc}
\usepackage{textcomp}
\usepackage[OT1, T1]{fontenc}
%\usepackage[francais]{babel}
\usepackage[frenchb]{babel}
\usepackage{hyperref}
%opening
\title{Exercice développement WEB}
\author{Yvan Maillot}
\date{Mardi 23 octobre 2018}

\begin{document}
\maketitle

\begin{enumerate}
 \item Ajouter une page pour créer et éditer des catégories (comme c'est fait pour les clients) et ajouter les catégories A, B et Z au lancement.
 \item Ajouter un menu sur toutes les pages pour basculer entre catégories et clients
 \item Faire que les catégories comme les clients s'affichent dans l'ordre alphabétique
 \item Un client appartient à 0 ou 1 catégorie, une catégorie est associée à 0, 1 ou plusieurs clients. Faire ce qu'il faut et ajouter la possibilité d'attribuer une catégorie à un client à sa création ou à son édition ou de la modifier.
 \item Modifier l'affichage de la liste des clients pour que leur catégorie s'affiche si elle existe.
 \item Ajouter la possibilité d'afficher tous les clients d'une catégorie par l'url. Par exemple \url{localhost:8080/customer/B} affiche la liste de tous les clients de la catégorie B.
\end{enumerate}



\end{document}
